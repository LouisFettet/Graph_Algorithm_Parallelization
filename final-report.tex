\documentclass[12pt]{article}
\usepackage{hyperref}
\title{\textbf{Graph Algorithm Parallelization}\\ 
Edmonds-Karp Algorithm Implementation in the Go Programming Language\\ 
Final Report}
\date{Computer Science Capstone Project Spring 2013}
\author{Louis L. Fettet}
\begin{document}
\maketitle
\section{Project Definition}
The initial goal of this project was to design a parallel implementation of the Edmonds-Karp algorithm, which is a modification of the Fords-Fulkerson used to calculate the maximum flow through a directed graph or flow network.  For all extensive purposes, that goal has been accomplished and no additional work is required.
\section{Design}
\indent The project was built and implemented in Google's Golang, which I had never used before now.  The source files of the project are separated into a main directory, used primarily for testing, and a graph package, where the bulk of the project was written.  The graph package is comprised primarily of:
\begin {enumerate}
\item \textbf{graph-struct.go:} houses the weighted graph structure and accompanying methods to create and edit graphs
\item \textbf{algorithm.go:} serial implementation of the Edmonds-Karp
\item \textbf{parallel-algo.go:}  implementation of the Edmonds-Karp that utilizes a Breadth-First Search which runs concurrent checks on possible paths
\item \textbf{test.go:} initialization of sample graphs to run time trials of the algorithms against each other in main
\end {enumerate}
\section{Results}
\indent Once completed, the algorithms were run on six small test cases and a randomly-generated graph of much larger size.  Interestingly enough, due to the overhead of implementing a semaphore and other channels, the concurrent algorithm ran slightly slower than the serial on the small graphs, but impressively faster on the larger one.\\
\section{Source Code}
All project resources can be found on my GitHub Capstone repository,\\*
\centerline{\url{http://github.com/LouisFettet/Graph_Algorithm_Parallelization}}
\end{document}